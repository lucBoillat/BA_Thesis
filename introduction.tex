\chapter{Introduction}
A financial service provider based in Zurich operates a bonus point system and its associated reward shop. When participants of the program make transactions using their credit cards, issued by the service provider, bonus points are awarded to the clients. The number of points a client receives depends on the amount of money they have spent. Collected points can be exchanged in the reward shop for products and coupons. This means that every merchant who wishes to sell its products on the reward shop has to contact the service provider and form a contract with him. The two main drawbacks of this approach are on one hand, the administrative effort on the provider's side which is needed to (1) maintain relations with merchants and (2) manage the reward shop, while on the other hand, the lack of awareness and interest of the point system by the clients due to its restricted nature.
To counter these drawbacks, the Bazo cryptocurrency has been developed, as a possible replacement for the traditional system. A blockchain-based, decentralized payment system that alleviates the provider's administrative costs by eliminating contracts with merchants. Using the currency Bazo Coin, the clients can buy products from merchants directly at their own Point-of-Sale, since the financial service provider is no longer the centralized record keeper of transactions and accounts. It is also possible for clients to transfer funds between each other. The only interaction between the service provider and merchants is the exchange of Bazo Coins for fiat currency. Similarly to the tradtitional system, Bazo will be invite-only, meaning only the clients of the service provider can interact with the blockchain, making Bazo a so-called private blockchain.

\section{Motivation}
To interact with the Bazo blockchain, two command-line applications are necessary: The Bazo Miner, which, together with all other Miners, runs the network, and the Bazo Client, which is mainly used to send transactions to the network. Every Bazo Miner stores all the blockchain and state data in its built-in storage component, however there is no way for a user to browse through and make use of that data using a GUI. Information about the health and productivity of the system are not available either. This is why a blockchain explorer is needed, a separate service that runs independently from the blockchain and lets users examine the blockchain data, without directly taking part in the network using miner or client applications.

\section{Description of Work}
This thesis documents the design, implementation and evaluation of a blockchain explorer for the private blockchain Bazo and its corresponding cryptocurrency Bazo Coin. The explorer allows users and administrators of Bazo to inspect and analyze the data making up the blockchain. Blocks, transactions and accounts are being displayed in an informative and well-structured manner, with the explorer acting as a visualizer for the blockchain. Statistical information about the blockchain will also be made available to the user. Furthermore the explorer features administator-only functionality, serving as a GUI for setting various system parameters from the web to the Bazo network.

\section{Thesis Outline}
Chapter 2 further explains the bazo blockchain in detail and analyzes existing blockchain explorers and statistics analysis platforms. Chapter 3 focuses on the design of the Bazo Blockchain Explorer, consisting of multiple components. Chapter 4 documents the implementation of the web application, followed by an evaluation in chapter 5. A summary and conclusions are presented in chapter 6.