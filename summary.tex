\chapter{Summary and Conclusions}

This paper documented the development of a blockchain explorer for the Bazo blockchain, heavily based on requirements elicited from an analysis of existing blockchain explorers and input from the financial service provider. The Bazo Blockchain Explorer allows users of the blockchain to inspect blockchain data through a graphical user interface, requiring no installation to use, since the application is available on the internet as a web app. Additionally, the operator of the system can set system parameters from the application, using an administrator panel. 

Explorers for the top two cryptocurrencies (at time of writing) were used for the analysis \cite{coinmarketcap}. Design decisions were then made, which included the structure of the system in relation to the existing Bazo infrastructure and concrete functionality of the application. Implementation-related topics such as high level software architecture and used technologies were documented in the chapter following design-decisions. Also included are code-snippets, that highlight certain algorithms crucial to the successful execution of the program. An evaluation of the developed application follows, discussing the performance and responsiveness of the web app, while working with a large amounts of data. Possible functionality that extends the current scope of the blockchain explorer is presented in the Future Works section.

The Bazo Blockchain Explorer makes the blockchain and its data transparent and accessible to all users. In conjunction with the Bazo PWA Payment System, users can now make transactions and check if they have been accepted by the system, without having to use command-line tools. This allows a test-run of the Bazo blockchain using real customers, since no programming knowledge is required. Due to the open-source nature of the blockchain explorer, anyone can host their own version of it, as long as the IP address of the bootstrap server is known.