\chapter*{Abstract}
\addcontentsline{toc}{chapter}{Abstract}

\selectlanguage{german}

Das Belohnungssystem eines Finanzdienstleisters besteht aus Bonuspunkten, welche beim gebrauch von Kredit- und Debitkarten gewonnen werden. Der Karteninhaber kann diese Punkte im Online-Shop des Dienstleisters gegen Waren und Gutscheine umtauschen. Dies bringt einen hohen administrativen Aufwand mit sich, da f�r jeden neuen H�ndler, welcher im Online-Shop seine Waren gegen Bonuspunkte verkaufen m�chte, ein speziell abgestimmter Vertrag erstellt werden muss. Zus�tzlich ist die Popularit�t des Services nicht erwartungsgem�ss, da die Punkte nur in diesem einen Online-Shop benutzbar sind. Zusammen mit der Universit�t Z�rich wurde darum die Bazo Kryptow�hrung entwickelt, welche eine dezentralisierte Verwaltung der Punkte und Konten erm�glicht. Dies hat den Vorteil, das H�ndler an ihrem eigenen PoS ihre Waren gegen Bazo Coins verkaufen k�nnen. Der einzige Kontakt, welcher die H�ndler mit dem Finanzdienstleister haben werden ist das Umtauschen von Bazo Coins in Fiat W�hrung.
Die Bazo Software besteht aus zwei Kommandozeilen-Tools welche die verarbeiteten Daten der Blockchain zwar speichern, jedoch nur bedingt dem Benutzer lesbar pr�sentieren. Diese Arbeit dokumentiert das Design, die Entwicklung und die Evaluation eines Blockchain Explorers f�r die Bazo Blockchain. Der Explorer erm�glicht dem Benutzer �ber einen Webbrowser die Blockchain-Daten zu durchsuchen und grafisch darzustellen. Ebenfalls verf�gt der Explorer �ber eine Benutzeroberfl�che f�r Administratoren, damit Systemparameter f�r die Blockchain gesetzt werden k�nnen.

\selectlanguage{english}
The reward system of a financial service provider consists of bonus points, which can be amassed by using credit- and debit-cards. These points can be exchanged for goods and coupons in the online reward shop of the service provider. This causes significant administrative overhead for the provider, since for every merchant that wants to sell its products in the reward shop, a tailored contract has to be made. Additionally, popularity of the shop is not as expected, due to the bonus points being only useable in this specific shop. Jointly with the University of Zurich, the Bazo cryptocurrency was developed to counter these disadvantages of the  bonus point system. This enables a decentralized management of points and accounts, and permits merchants to sell their products at their own PoS for Bazo Coins. The only contact merchants now have with the financial service provider, is when they exchange their amassed Bazo Coins for fiat money.
The Bazo software consists of two command-line interfaces which, by design, save the processed data of the blockchain. However only limited access to this data is possible. This thesis covers the design, development and evaluation of a blockchain explorer for the Bazo cryptocurrency. The blockchain explorer enables users to display and browse through the blockchain data via a web-browser. Additionally, the explorer contains an admin-panel, where administrators of the system can set certain system parameters of the blockchain.